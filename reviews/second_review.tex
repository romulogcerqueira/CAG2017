\documentclass{article}
\usepackage{a4wide}
\usepackage{todonotes}
\usepackage[utf8]{inputenc}
\usepackage[T1]{fontenc}
\usepackage{amsmath}
\usepackage{graphicx}

\title{Response to the Reviewer's Comments}
\begin{document}

\maketitle

We thank the reviewers for their insightful and useful feedback. We have revised our paper in light of their comments. Please, find below our replies and comments about each of the issues raised, and how we modified the paper to address them. Beyond the suggested reviews, we also made changes in some parts of the paper to adhere to the suggestions, turning the paper more coherent and cohesive. To make clear where we modified, all changes in the text are in blue in this revised version.

\begin{verbatim}
>> REVIEWER #1 <<

> The required improvement in the article was done. The only point that I am
> still waiting an answer is about the validation of the simulator. The authors
> argue that they have the real data and the the scenario but they will just do
> this in the next months. I think it could improve the impact and the
> acceptance of the work.

\end{verbatim}

\textbf{XXXXXXXXXXX}

\begin{verbatim}

>> REVIEWER #2 <<

> The new version of the paper improved all my comments and suggestions, from my
> first review. I agree with its acceptance.

\end{verbatim}

\textbf{We thank for your feedback. We have addressed the other reviewers' comments in the revised version, and we hope your expectations continue to be satisfied.}

\begin{verbatim}

>> REVIEWER #3 <<

> Graphical Abstract
> ==================
> Ok.
>
> Highlights
> ==========
> Ok, except for the item "Scenarios produced more realistically than other
> simulators". "Realism" is discussed along the review.
>
> Paper Review
> ============
>
> The present work proposes a method for the simulation of sonar sensors.
> Differently from previous methods, the proposed technique takes advantage
> of the GPU to achieve realtime performance and is able to reproduce the
> operation of FLS and MSIS sonars sensors.
>
> Text
> ====
>
> The text still contains confusing passages that demand review. For instance:
>
> -- Abstract
> "For that, simulation of sonar (...)."
> "This proposed system (...)."
>
> ---> The beginning of those sentences sound strange.
>
> -- Abstract
> "Our system exploits the rasterization pipeline in order to simulate the sonar
> devices, which are rendered by three parameters (...)."
>
> --> It looks like that the "sonar devices" are rendered. It would better to
> rephrase that.
>
> -- lines 72-74
> "(...) we take advantage of precomputed geometric data during the rasterization
> pipeline to compute the acoustic frame (...)"
>
> ---> Confusing sentence. What is "geometric data"? I believe that, in this case,
> "geometric data" refers to the fragments generated by the rasterization stage.
> Thus, instead of "geometric data", maybe the word "samples" would be more
> appropriate.
>
> -- lines 109-110
> "(...) until they hit any object or be completely absorbed."
>
> --> "(...) or are completely (...)."
>
> -- lines 192-204
> The paragraph in lines 192-204 describes some technicalities, such as file
> formats (related to the use of the Rock-Gazebo framework), which are completely
> detached from the core discussion proposed in the paper. That discussion could
> be eventually removed.
>
> -- lines 265-267
> "Normal mapping is a perturbation rendering technique to simulate wrinkles on
> the object surface by passing textures (...)."
>
> ---> "Perturbation rendering technique to simulate...", I've found this sentence
> a bit confusing since I have never seen the expression "perturbation rendering
> technique" associated to the normal mapping.
>
> ---> What does it mean "by passing textures"?
>
> -- lines 404-405
> "(...) in order to provid enough (...)"
> ---> "provide"
>
> -- lines 457-458
> "A GPU-based simulator for imaging sonar simulation was presented here."
> ---> Confusing.
>
> Contributions and limitations
> =============================
>
> The abstract highlights the importance of simulating sonar sensors
> and introduce the new proposed method. However, once again, nothing is said
> about the difficulties involved in such a simulation, except for the high
> computational cost involved. Again, no mentions about the general limitations
> of the current methods and about the insights that led to the proposed method.
>
> From the abstract, it is not possible to say what sets the proposed method apart
> from the existing ones. For instance, in one sentence it is said that "Our
> system exploits the rasterization pipeline (...)". From this sentence, one
> cannot say if previous methods do, or don't, use the rasterization pipeline. If
> the exploitation of the rasterization pipeline is an exclusive feature of the
> new method, it should be stated. For instance "Differently from previous
> methods, the proposed technique exploits the rasterization pipeline (...)". In
> this case, it would be evident the difference with respect to previous works.
>
> -- abstract
> "(...) as well as generating realistic sonar image quality in different virtual
> underwater scenarios."
>
> ---> Realistic in which sense? Is it just visually convincent or numerically
> accurate?
>
> The method
> ==========
>
> The authors have made a good job in rewritting the method and its implementation.
> It is, actually, a straightforward method, which employs a technique very
> similar to that of deferred shading. The main point, in this context, is the
> computation of the intensity values for the bins, which seems to be a
> contribution.
>
> Despite the method developed to compute the intensity of the bins, which seems
> to be new, and the general improvements in the structure of the paper, the
> general contribution of the paper is still diffuse for me. From
> the text, it seems that the premisse of the work was to develop a realtime sonar
> simulator targeted at interactive applications. New features would be
> incorporated into the simulator only if they would not affect interactivity.
> Thus, important effects, such as the reverberation, were not included because of
> their higher computational demands. The end result is a simulator that is fast,
> but that lacks accuracy.
>
> Usually, when a existing application is mapped to higher performance hardware,
> such as a GPU, it is expected the application to have its performance improved
> while still maintaining its core features. However, in this case, despite the
> use of a GPU, performance was improved at the cost of the simplification of the
> application.
>
> Decision
> ========
>
> I think that this work, in its current state, has potential to be accepted in a
> conference. It would be certainly a good opportunity to show the work to a
> broader audience and to start the discussion. However, for acceptance in a
> journal, I would expect a more sound contribution. For instance, the impact of
> this work would increase a lot if it would be able to handle reverberation
> and if some kind of validation could be also provided.
>
> However, in its current state, and from all that was previously discussed,
> I've decided for its rejection.

\end{verbatim}

\textbf{XXXXXXXXXXXXXX}

\begin{verbatim}

>> REVIEWER #4 <<

> As previously commented, the document is very well-written and organized,
> theoretically well supported, with a clear presentation and the experiments
> are sufficient to prove the system effectiveness.
> Although I believe this paper is ready to be published, satisfactory
> responses should be presented to the other reviewers' concerns before the
> final acceptance.

\end{verbatim}

% \todo[inline, color=blue!40]{Undefined responsible's answer}
\textbf{We thank for your feedback. We have addressed the other reviewers' comments in the revised version, and we hope your expectations continue to be satisfied.}

\end{document}

\grid
\grid
